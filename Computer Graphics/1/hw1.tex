%%%%%%%%%%%%%%%%%%%%%%%%%%%%%%%%%%%%%%%%%
% Structured General Purpose Assignment
% LaTeX Template
%
% This template has been downloaded from:
% http://www.latextemplates.com
%
% Original author:
% Ted Pavlic (http://www.tedpavlic.com)
%
% Note:
% The \lipsum[#] commands throughout this template generate dummy text
% to fill the template out. These commands should all be removed when 
% writing assignment content.
%
%%%%%%%%%%%%%%%%%%%%%%%%%%%%%%%%%%%%%%%%%

%----------------------------------------------------------------------------------------
%	PACKAGES AND OTHER DOCUMENT CONFIGURATIONS
%----------------------------------------------------------------------------------------

\documentclass{article}

\usepackage{fancyhdr} % Required for custom headers
\usepackage{lastpage} % Required to determine the last page for the footer
\usepackage{extramarks} % Required for headers and footers
\usepackage{graphicx} % Required to insert images
\usepackage{lipsum} % Used for inserting dummy 'Lorem ipsum' text into the template
\usepackage{amsmath}
\usepackage{hyperref}

% Margins
\topmargin=-0.45in
\evensidemargin=0in
\oddsidemargin=0in
\textwidth=6.5in
\textheight=9.0in
\headsep=0.25in 

\linespread{1.1} % Line spacing

% Set up the header and footer
\pagestyle{fancy}
\lhead{\hmwkAuthorName} % Top left header
\chead{\hmwkClass\ (\hmwkClassInstructor\ \hmwkClassTime): \hmwkTitle} % Top center header
\rhead{\firstxmark} % Top right header
\lfoot{\lastxmark} % Bottom left footer
\cfoot{} % Bottom center footer
\rfoot{Page\ \thepage\ of\ \pageref{LastPage}} % Bottom right footer
\renewcommand\headrulewidth{0.4pt} % Size of the header rule
\renewcommand\footrulewidth{0.4pt} % Size of the footer rule

\setlength\parindent{0pt} % Removes all indentation from paragraphs

%----------------------------------------------------------------------------------------
%	DOCUMENT STRUCTURE COMMANDS
%	Skip this unless you know what you're doing
%----------------------------------------------------------------------------------------

% Header and footer for when a page split occurs within a problem environment
\newcommand{\enterProblemHeader}[1]{
\nobreak\extramarks{#1}{#1 continued on next page\ldots}\nobreak
\nobreak\extramarks{#1 (continued)}{#1 continued on next page\ldots}\nobreak
}

% Header and footer for when a page split occurs between problem environments
\newcommand{\exitProblemHeader}[1]{
\nobreak\extramarks{#1 (continued)}{#1 continued on next page\ldots}\nobreak
\nobreak\extramarks{#1}{}\nobreak
}

\setcounter{secnumdepth}{0} % Removes default section numbers
\newcounter{homeworkProblemCounter} % Creates a counter to keep track of the number of problems

\newcommand{\homeworkProblemName}{}
\newenvironment{homeworkProblem}[1][Problem \arabic{homeworkProblemCounter}]{ % Makes a new environment called homeworkProblem which takes 1 argument (custom name) but the default is "Problem #"
\stepcounter{homeworkProblemCounter} % Increase counter for number of problems
\renewcommand{\homeworkProblemName}{#1} % Assign \homeworkProblemName the name of the problem
\section{\homeworkProblemName} % Make a section in the document with the custom problem count
\enterProblemHeader{\homeworkProblemName} % Header and footer within the environment
}{
\exitProblemHeader{\homeworkProblemName} % Header and footer after the environment
}

\newcommand{\problemAnswer}[1]{ % Defines the problem answer command with the content as the only argument
\noindent\framebox[\columnwidth][c]{\begin{minipage}{0.98\columnwidth}#1\end{minipage}} % Makes the box around the problem answer and puts the content inside
}

\newcommand{\homeworkSectionName}{}
\newenvironment{homeworkSection}[1]{ % New environment for sections within homework problems, takes 1 argument - the name of the section
\renewcommand{\homeworkSectionName}{#1} % Assign \homeworkSectionName to the name of the section from the environment argument
\subsection{\homeworkSectionName} % Make a subsection with the custom name of the subsection
\enterProblemHeader{\homeworkProblemName\ [\homeworkSectionName]} % Header and footer within the environment
}{
\enterProblemHeader{\homeworkProblemName} % Header and footer after the environment
}
   
%----------------------------------------------------------------------------------------
%	NAME AND CLASS SECTION
%----------------------------------------------------------------------------------------

\newcommand{\hmwkTitle}{Assignment\ \#1} % Assignment title
\newcommand{\hmwkDueDate}{Monday,\ September\ 7,\ 2017} % Due date
\newcommand{\hmwkClass}{CSCI\ 6964} % Course/class
\newcommand{\hmwkClassTime}{4:00pm} % Class/lecture time
\newcommand{\hmwkClassInstructor}{Franklin} % Teacher/lecturer
\newcommand{\hmwkAuthorName}{Clayton Rayment} % Your name

%----------------------------------------------------------------------------------------
%	TITLE PAGE
%----------------------------------------------------------------------------------------

\title{
\vspace{2in}
\textmd{\textbf{\hmwkClass:\ \hmwkTitle}}\\
\normalsize\vspace{0.1in}\small{Due\ on\ \hmwkDueDate}\\
\vspace{3in}
}

\author{\textbf{\hmwkAuthorName}}
\date{} % Insert date here if you want it to appear below your name

%----------------------------------------------------------------------------------------

\begin{document}

\maketitle

%----------------------------------------------------------------------------------------
%	TABLE OF CONTENTS
%----------------------------------------------------------------------------------------

%\setcounter{tocdepth}{1} % Uncomment this line if you don't want subsections listed in the ToC

\newpage
\tableofcontents
\newpage

%----------------------------------------------------------------------------------------
%	PROBLEM 1
%----------------------------------------------------------------------------------------

% To have just one problem per page, simply put a \clearpage after each problem

\begin{homeworkProblem}
    One graphics pioneer was Ivan Sutherland. Name an influential tool that he created and an influential algorithm that he helped create.\vspace{10pt} % Question
    
    \problemAnswer{ % Answer
        % \includegraphics[width=0.75\columnwidth]{example_figure} % Example image
        Ivan Sutherland was responsible for the creation of Sketchpad, which was an early predecessor to GUIs as we know them today. Sketchpad contained an algorithm developed by Ivan Sutherland which allowed for efficient calculation of clipping between objects.
    }

\end{homeworkProblem}

%----------------------------------------------------------------------------------------
%	PROBLEM 2
%----------------------------------------------------------------------------------------

\begin{homeworkProblem} % Custom section title
    Consider these 3-D vectors: A=(0,4,2), B=(1,2,3), C=(8,7,9). Compute:\vspace{10pt}

    %--------------------------------------------

    \begin{homeworkSection}{(a)} % Section within problem
        $A \cdot (B \times C)$\vspace{10pt} % Question

        \problemAnswer{ % Answer
            \begin{align*}
                B \times C &= (1, 2, 3) \times (8, 7, 9)\\
                &=\begin{vmatrix}
                    i & j & k\\
                    1 & 2 & 3\\
                    8 & 7 & 9\\
                \end{vmatrix}\\
                &= (-3, 15, -9)\\
                A \cdot (B \times C) &= (0, 4, 2) \cdot (-3, 15, -9)\\
                &= 0 + 60 - 18\\
                &= 42\\
            \end{align*}
        }
    \end{homeworkSection}

    %--------------------------------------------

    \begin{homeworkSection}{(b)} % Section within problem
        $(A \times B) \cdot C$\vspace{10pt}

        \problemAnswer{ % Answer
            \begin{align*}
                A \times B &= (0, 4, 2) \times (1, 2, 3)\\
                &=\begin{vmatrix}
                    i & j & k\\
                    0 & 4 & 2\\
                    1 & 2 & 3\\
                \end{vmatrix}\\
                &= (8, 2, -4)\\
                (A \times B) \cdot C &= (8, 2, -4) \cdot (8, 7, 9)\\
                &= 64 + 14 - 36\\
                &= 42\\
            \end{align*}
        }
    \end{homeworkSection}

%--------------------------------------------

\end{homeworkProblem}

%----------------------------------------------------------------------------------------
%	PROBLEM 3
%----------------------------------------------------------------------------------------

\begin{homeworkProblem} % Custom section title
    Suppose that we have a plane in 3-D through the points A(3,2,0), B(2,2,0), and C(0,1,1).\vspace{10pt}

    %--------------------------------------------

    \begin{homeworkSection}{(a)} % Section within problem
        What is its equation, in the form $ax+by+cz+d=0$?\vspace{10pt} % Question

        \problemAnswer{ % Answer
            \begin{align*}
                \hat{n} &= AB \times AC\\
               AB \times AC &=
                \begin{vmatrix}
                    i & j & k\\
                    -1 & 0 & 0\\
                    -3 & -1 & 1\\
                \end{vmatrix}\\
                &= (0, 1, 1)\\
               &0(x-3) + 1(y-2) + 1(z-0)\\
               &y+z-2 = 0\\
            \end{align*}
        }
    \end{homeworkSection}

    %--------------------------------------------

    \begin{homeworkSection}{(b)} % Section within problem
        Consider the line L thru the points O(0,0,0) and P(1,1,1). Where does this line intersect the plane?\vspace{10pt}

        \problemAnswer{ % Answer
            \begin{align*}
                P_x(t) &= 1 * t\\
                P_y(t) &= 1 * t\\
                P_z(t) &= 1 * t\\
                \text{Substitution: }\\
                (t) + (t) - 2 &= 0\\
                t &= 1\\
                \text{Intersection: }\\
                P_i &= (1, 1, 1)\\
            \end{align*}
        }
    \end{homeworkSection}

    %--------------------------------------------

\end{homeworkProblem}
%--------------------------------------------

%----------------------------------------------------------------------------------------
%	PROBLEM 4
%----------------------------------------------------------------------------------------

\begin{homeworkProblem}
    Modify the program in http://www.cs.unm.edu/~angel/WebGL/7E/CLASS/square.html , which calls several other files, to display shapes that are the first initials of the last names of the team members.\vspace{10pt}

    \problemAnswer{ % Answer
        The solution for this homework can be seen at: \url{https://www.rpi.edu/~STUDENT/homeworks/hw1/initial.html}\\
        Changes were made mostly to square.js, and it was renamed initial.js, so the square.html file had to be changed to reference the new file, and was also updated to match the naming scheme.
        
    }
\end{homeworkProblem}

%----------------------------------------------------------------------------------------

%----------------------------------------------------------------------------------------
%	PROBLEM 5
%----------------------------------------------------------------------------------------

\begin{homeworkProblem}
    Do exercise 1.1 from the textbook, page 37.\vspace{10pt}
    
    \problemAnswer{ % Answer
        The main advantage to using the non-physical graphics pipeline approach is to improve speed of calculation, making image display to the user real-time. The main disadvantage to this, however is that this requires special hardware, which is more complicated than non-pipelined hardware.
    }
\end{homeworkProblem}
    
%----------------------------------------------------------------------------------------


%----------------------------------------------------------------------------------------
%	PROBLEM 6
%----------------------------------------------------------------------------------------

\begin{homeworkProblem}
    Do exercise 1.8 from the textbook, page 38.\vspace{10pt}

    \begin{homeworkSection}{(a)}
        1280 x 1024 pixels at 72Hz\\
        \problemAnswer{
            \begin{align*}
                1280 \text{pixels} \times 1024 \text{pixels} \times 24 \frac{\text{bits}}{\text{pixel}} &= 31457280 \text{ bits }\\
                31457280 \text{ bits } * 72 \text{sec}^{-1} &= 2264924160 \frac{\text{bits}}{\text{sec}}\\
                &= 2160 \frac{\text{Mib}}{\text{sec}}\\
            \end{align*}
        }
    \end{homeworkSection}

    \begin{homeworkSection}{(a)}
        640 x 480 pixels at 60Hz Interlaced\\
        \problemAnswer{
            Interlaced video works with the same horizontal resolution, but at an effective half vertical resolution. The calculation then becomes:
            \begin{align*}
                640 \text{pixels} \times 240 \text{pixels} \times 24 \frac{\text{bits}}{\text{pixel}} &= 3686400 \text{ bits }\\
                3686400 \text{ bits } * 60 \text{sec}^{-1} &= 221184000 \frac{\text{bits}}{\text{sec}}\\
                &\approx 211 \frac{\text{Mib}}{\text{sec}}\\
            \end{align*}
        }
    \end{homeworkSection}
\end{homeworkProblem}
    
%----------------------------------------------------------------------------------------

\end{document}
