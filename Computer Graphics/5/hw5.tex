%%%%%%%%%%%%%%%%%%%%%%%%%%%%%%%%%%%%%%%%%
% Structured General Purpose Assignment
% LaTeX Template
%
% This template has been downloaded from:
% http://www.latextemplates.com
%
% Original author:
% Ted Pavlic (http://www.tedpavlic.com)
%
% Note:
% The \lipsum[#] commands throughout this template generate dummy text
% to fill the template out. These commands should all be removed when 
% writing assignment content.
%
%%%%%%%%%%%%%%%%%%%%%%%%%%%%%%%%%%%%%%%%%

%----------------------------------------------------------------------------------------
%	PACKAGES AND OTHER DOCUMENT CONFIGURATIONS
%----------------------------------------------------------------------------------------

\documentclass{article}

\usepackage{fancyhdr} % Required for custom headers
\usepackage{lastpage} % Required to determine the last page for the footer
\usepackage{extramarks} % Required for headers and footers
\usepackage{graphicx} % Required to insert images
\usepackage{lipsum} % Used for inserting dummy 'Lorem ipsum' text into the template
\usepackage{amsmath}
\usepackage{hyperref}

% Margins
\topmargin=-0.45in
\evensidemargin=0in
\oddsidemargin=0in
\textwidth=6.5in
\textheight=9.0in
\headsep=0.25in 

\linespread{1.1} % Line spacing

% Set up the header and footer
\pagestyle{fancy}
\lhead{\hmwkAuthorName} % Top left header
\chead{\hmwkClass\ (\hmwkClassInstructor\ \hmwkClassTime): \hmwkTitle} % Top center header
\rhead{\firstxmark} % Top right header
\lfoot{\lastxmark} % Bottom left footer
\cfoot{} % Bottom center footer
\rfoot{Page\ \thepage\ of\ \pageref{LastPage}} % Bottom right footer
\renewcommand\headrulewidth{0.4pt} % Size of the header rule
\renewcommand\footrulewidth{0.4pt} % Size of the footer rule

\setlength\parindent{0pt} % Removes all indentation from paragraphs

%----------------------------------------------------------------------------------------
%	DOCUMENT STRUCTURE COMMANDS
%	Skip this unless you know what you're doing
%----------------------------------------------------------------------------------------

% Header and footer for when a page split occurs within a problem environment
\newcommand{\enterProblemHeader}[1]{
\nobreak\extramarks{#1}{#1 continued on next page\ldots}\nobreak
\nobreak\extramarks{#1 (continued)}{#1 continued on next page\ldots}\nobreak
}

% Header and footer for when a page split occurs between problem environments
\newcommand{\exitProblemHeader}[1]{
\nobreak\extramarks{#1 (continued)}{#1 continued on next page\ldots}\nobreak
\nobreak\extramarks{#1}{}\nobreak
}

\setcounter{secnumdepth}{0} % Removes default section numbers
\newcounter{homeworkProblemCounter} % Creates a counter to keep track of the number of problems

\newcommand{\homeworkProblemName}{}
\newenvironment{homeworkProblem}[1][Problem \arabic{homeworkProblemCounter}]{ % Makes a new environment called homeworkProblem which takes 1 argument (custom name) but the default is "Problem #"
\stepcounter{homeworkProblemCounter} % Increase counter for number of problems
\renewcommand{\homeworkProblemName}{#1} % Assign \homeworkProblemName the name of the problem
\section{\homeworkProblemName} % Make a section in the document with the custom problem count
\enterProblemHeader{\homeworkProblemName} % Header and footer within the environment
}{
\exitProblemHeader{\homeworkProblemName} % Header and footer after the environment
}

\newcommand{\problemAnswer}[1]{ % Defines the problem answer command with the content as the only argument
\noindent\framebox[\columnwidth][c]{\begin{minipage}{0.98\columnwidth}#1\end{minipage}} % Makes the box around the problem answer and puts the content inside
}

\newcommand{\homeworkSectionName}{}
\newenvironment{homeworkSection}[1]{ % New environment for sections within homework problems, takes 1 argument - the name of the section
\renewcommand{\homeworkSectionName}{#1} % Assign \homeworkSectionName to the name of the section from the environment argument
\subsection{\homeworkSectionName} % Make a subsection with the custom name of the subsection
\enterProblemHeader{\homeworkProblemName\ [\homeworkSectionName]} % Header and footer within the environment
}{
\enterProblemHeader{\homeworkProblemName} % Header and footer after the environment
}
   
%----------------------------------------------------------------------------------------
%	NAME AND CLASS SECTION
%----------------------------------------------------------------------------------------

\newcommand{\hmwkTitle}{Assignment\ \#5} % Assignment title
\newcommand{\hmwkDueDate}{Monday,\ October\ 16,\ 2017} % Due date
\newcommand{\hmwkClass}{CSCI\ 6964} % Course/class
\newcommand{\hmwkClassTime}{4:00pm} % Class/lecture time
\newcommand{\hmwkClassInstructor}{Franklin} % Teacher/lecturer
\newcommand{\hmwkAuthorName}{Clayton Rayment} % Your name

%----------------------------------------------------------------------------------------
%	TITLE PAGE
%----------------------------------------------------------------------------------------

\title{
\vspace{2in}
\textmd{\textbf{\hmwkClass:\ \hmwkTitle}}\\
\normalsize\vspace{0.1in}\small{Due\ on\ \hmwkDueDate}\\
\vspace{3in}
}

\author{\textbf{\hmwkAuthorName}}
\date{} % Insert date here if you want it to appear below your name

%----------------------------------------------------------------------------------------

\begin{document}

\maketitle

%----------------------------------------------------------------------------------------
%	TABLE OF CONTENTS
%----------------------------------------------------------------------------------------

%\setcounter{tocdepth}{1} % Uncomment this line if you don't want subsections listed in the ToC

\newpage
\tableofcontents
\newpage

%----------------------------------------------------------------------------------------
%	PROBLEM 1
%----------------------------------------------------------------------------------------

% To have just one problem per page, simply put a \clearpage after each problem

\begin{homeworkProblem}
    What is the angle (in degrees) between these two vectors: (1,2,0), (1,2,3)?\\
    \problemAnswer{ % Answer
        \begin{align*}
            \cos\theta &= \frac{\vec{u}\cdot\vec{v}}{\|\vec{u}\|\cdot\|\vec{v}\|}\\
            \cos\theta &= \frac{5}{\sqrt{5}\cdot\sqrt{14}}\\
            \cos\theta &= \frac{5}{\sqrt{70}}\\
            \theta &= \cos^{-1}\frac{5}{\sqrt{70}}\\
            \theta &\approx 53^\circ
        \end{align*}
    }
\end{homeworkProblem}

%----------------------------------------------------------------------------------------
%	PROBLEM 2
%----------------------------------------------------------------------------------------

% To have just one problem per page, simply put a \clearpage after each problem

\begin{homeworkProblem}
    (Reverse engineering rotations) In 2D, if the point (4,2) rotates about the origin to (2,-4), what's the angle?\\
    \problemAnswer{ % Answer
    }
\end{homeworkProblem}

%----------------------------------------------------------------------------------------
%	PROBLEM 3
%----------------------------------------------------------------------------------------

% To have just one problem per page, simply put a \clearpage after each problem

\begin{homeworkProblem}
    Give the matrix M that has this property: for all vectors p:
    \begin{align*}
        Mp = \begin{pmatrix}2\\4\\5\end{pmatrix} \times p
    \end{align*}
    \problemAnswer{ % Answer
    }
\end{homeworkProblem}

%----------------------------------------------------------------------------------------
%	PROBLEM 4
%----------------------------------------------------------------------------------------

% To have just one problem per page, simply put a \clearpage after each problem

\begin{homeworkProblem}
    Give the matrix M that has this property: for all vectors p:
    \begin{align*}
        Mp = \left( \begin{pmatrix}2\\4\\5\end{pmatrix} \cdot p \right) \begin{pmatrix}2\\4\\5\end{pmatrix}
    \end{align*}
    \problemAnswer{ % Answer
    }
\end{homeworkProblem}

%----------------------------------------------------------------------------------------
%	PROBLEM 5
%----------------------------------------------------------------------------------------

% To have just one problem per page, simply put a \clearpage after each problem

\begin{homeworkProblem}
    Why can the following not possibly be a 3D Cartesian rotation matrix?
    \begin{align*}
        \begin{pmatrix}  3& 0 &0\\1 & 0 &0\\0& 0 &1\end{pmatrix}
    \end{align*}
    \problemAnswer{ % Answer
        The above matrix can not be a 3d cartesian rotation matrix because the determinant is zero, and to be a valid rotation matrix, the determinant must be 1.
    }
\end{homeworkProblem}

%----------------------------------------------------------------------------------------
%	PROBLEM 6
%----------------------------------------------------------------------------------------

% To have just one problem per page, simply put a \clearpage after each problem

\begin{homeworkProblem}
    Use any method (not involving soliciting answers on the internet) to rotate the point (4,4,6) by 120 degrees about the axis (2,2,3). Explain your method. (E.g., if you saw the answer in a vision, are your visions generally accurate?)\\
    \problemAnswer{ % Answer
    }
\end{homeworkProblem}

%----------------------------------------------------------------------------------------
%	PROBLEM 7
%----------------------------------------------------------------------------------------

% To have just one problem per page, simply put a \clearpage after each problem

\begin{homeworkProblem}
    Can the volume of a small cube change when its vertices are rotated? (yes or no). Why (not)?\\
    \problemAnswer{ % Answer
        No, rotation is a uniform operation, so each vertex will be rotated by an amount proportional to how far they are away from the axis of rotation. Another way to think of this: Should you rotate the reference frame, the volume would not change, therefore, rotating the cube will not change the volume.
    }
\end{homeworkProblem}

%----------------------------------------------------------------------------------------
%	PROBLEM 8
%----------------------------------------------------------------------------------------

% To have just one problem per page, simply put a \clearpage after each problem

\begin{homeworkProblem}
    What is the Event Loop?\\
    \problemAnswer{ % Answer
        The Event Loop is a feature of JS which allows for asynchronous calls to other functions. Once a user calls a function, it is entered onto a call stack. Once the function returns, the value is given back to the user.
    }
\end{homeworkProblem}


%----------------------------------------------------------------------------------------
%	PROBLEM 9
%----------------------------------------------------------------------------------------

% To have just one problem per page, simply put a \clearpage after each problem

\begin{homeworkProblem}
    Why does putting all your vertices into an array and telling OpenGL about it make a big graphics program faster?\\
    \problemAnswer{ % Answer
        Putting your vertices next to eachother in an array reduces cache missing because you have contiguous data. This reduces the number of miss penalties that the compute device must suffer.
    }
\end{homeworkProblem}


%----------------------------------------------------------------------------------------
%	PROBLEM 10
%----------------------------------------------------------------------------------------

% To have just one problem per page, simply put a \clearpage after each problem

\begin{homeworkProblem}
    Since the Z (aka depth) buffer looks so useful, why is it not enabled by default?\\
    \problemAnswer{ % Answer
        The depth buffer is disabled by default in order to use less resources on the compute device.
    }
\end{homeworkProblem}

%----------------------------------------------------------------------------------------
%	PROBLEM 11
%----------------------------------------------------------------------------------------

% To have just one problem per page, simply put a \clearpage after each problem

\begin{homeworkProblem}
    What's the quaternion representing a rotation of 180 degrees about the axis (0,1,0)?\\
    \problemAnswer{ % Answer
        -1
    }
\end{homeworkProblem}

%----------------------------------------------------------------------------------------
%	PROBLEM 12
%----------------------------------------------------------------------------------------

% To have just one problem per page, simply put a \clearpage after each problem

\begin{homeworkProblem}
    Use the quaternion formulation to rotate the point (0,1,0) by 180 degrees about the axis (0,1,0).\\
    \problemAnswer{ % Answer
    }
\end{homeworkProblem}

%----------------------------------------------------------------------------------------
%	PROBLEM 13
%----------------------------------------------------------------------------------------

% To have just one problem per page, simply put a \clearpage after each problem

\begin{homeworkProblem}
    Use the vector formulation to rotate the point (0,1,0) by 180 degrees about the axis (0,1,0).\\
    \problemAnswer{ % Answer
    }
\end{homeworkProblem}

%----------------------------------------------------------------------------------------
%	PROBLEM 14
%----------------------------------------------------------------------------------------

% To have just one problem per page, simply put a \clearpage after each problem

\begin{homeworkProblem}
    Extend your program from last week that displays the Starship Enterprise as follows:
    \begin{enumerate}
        \item Do the rotation in the vertex shader instead of in the javascript program.
        \item Make the color of each pixel depend on its z-value.
    \end{enumerate}
    \problemAnswer{ % Answer
    }
\end{homeworkProblem}


\end{document}

%----------------------------------------------------------------------------------------