%%%%%%%%%%%%%%%%%%%%%%%%%%%%%%%%%%%%%%%%%
% Structured General Purpose Assignment
% LaTeX Template
%
% This template has been downloaded from:
% http://www.latextemplates.com
%
% Original author:
% Ted Pavlic (http://www.tedpavlic.com)
%
% Note:
% The \lipsum[#] commands throughout this template generate dummy text
% to fill the template out. These commands should all be removed when 
% writing assignment content.
%
%%%%%%%%%%%%%%%%%%%%%%%%%%%%%%%%%%%%%%%%%

%----------------------------------------------------------------------------------------
%	PACKAGES AND OTHER DOCUMENT CONFIGURATIONS
%----------------------------------------------------------------------------------------

\documentclass{article}

\usepackage{fancyhdr} % Required for custom headers
\usepackage{lastpage} % Required to determine the last page for the footer
\usepackage{extramarks} % Required for headers and footers
\usepackage{graphicx} % Required to insert images
\usepackage{lipsum} % Used for inserting dummy 'Lorem ipsum' text into the template
\usepackage{amsmath}
\usepackage{hyperref}

% Margins
\topmargin=-0.45in
\evensidemargin=0in
\oddsidemargin=0in
\textwidth=6.5in
\textheight=9.0in
\headsep=0.25in 

\linespread{1.1} % Line spacing

% Set up the header and footer
\pagestyle{fancy}
\lhead{\hmwkAuthorName} % Top left header
\chead{\hmwkClass\ (\hmwkClassInstructor\ \hmwkClassTime): \hmwkTitle} % Top center header
\rhead{\firstxmark} % Top right header
\lfoot{\lastxmark} % Bottom left footer
\cfoot{} % Bottom center footer
\rfoot{Page\ \thepage\ of\ \pageref{LastPage}} % Bottom right footer
\renewcommand\headrulewidth{0.4pt} % Size of the header rule
\renewcommand\footrulewidth{0.4pt} % Size of the footer rule

\setlength\parindent{0pt} % Removes all indentation from paragraphs

%----------------------------------------------------------------------------------------
%	DOCUMENT STRUCTURE COMMANDS
%	Skip this unless you know what you're doing
%----------------------------------------------------------------------------------------

% Header and footer for when a page split occurs within a problem environment
\newcommand{\enterProblemHeader}[1]{
\nobreak\extramarks{#1}{#1 continued on next page\ldots}\nobreak
\nobreak\extramarks{#1 (continued)}{#1 continued on next page\ldots}\nobreak
}

% Header and footer for when a page split occurs between problem environments
\newcommand{\exitProblemHeader}[1]{
\nobreak\extramarks{#1 (continued)}{#1 continued on next page\ldots}\nobreak
\nobreak\extramarks{#1}{}\nobreak
}

\setcounter{secnumdepth}{0} % Removes default section numbers
\newcounter{homeworkProblemCounter} % Creates a counter to keep track of the number of problems

\newcommand{\homeworkProblemName}{}
\newenvironment{homeworkProblem}[1][Problem \arabic{homeworkProblemCounter}]{ % Makes a new environment called homeworkProblem which takes 1 argument (custom name) but the default is "Problem #"
\stepcounter{homeworkProblemCounter} % Increase counter for number of problems
\renewcommand{\homeworkProblemName}{#1} % Assign \homeworkProblemName the name of the problem
\section{\homeworkProblemName} % Make a section in the document with the custom problem count
\enterProblemHeader{\homeworkProblemName} % Header and footer within the environment
}{
\exitProblemHeader{\homeworkProblemName} % Header and footer after the environment
}

\newcommand{\problemAnswer}[1]{ % Defines the problem answer command with the content as the only argument
\noindent\framebox[\columnwidth][c]{\begin{minipage}{0.98\columnwidth}#1\end{minipage}} % Makes the box around the problem answer and puts the content inside
}

\newcommand{\homeworkSectionName}{}
\newenvironment{homeworkSection}[1]{ % New environment for sections within homework problems, takes 1 argument - the name of the section
\renewcommand{\homeworkSectionName}{#1} % Assign \homeworkSectionName to the name of the section from the environment argument
\subsection{\homeworkSectionName} % Make a subsection with the custom name of the subsection
\enterProblemHeader{\homeworkProblemName\ [\homeworkSectionName]} % Header and footer within the environment
}{
\enterProblemHeader{\homeworkProblemName} % Header and footer after the environment
}
   
%----------------------------------------------------------------------------------------
%	NAME AND CLASS SECTION
%----------------------------------------------------------------------------------------

\newcommand{\hmwkTitle}{Assignment\ \#3} % Assignment title
\newcommand{\hmwkDueDate}{Monday,\ September\ 25,\ 2017} % Due date
\newcommand{\hmwkClass}{CSCI\ 6964} % Course/class
\newcommand{\hmwkClassTime}{4:00pm} % Class/lecture time
\newcommand{\hmwkClassInstructor}{Franklin} % Teacher/lecturer
\newcommand{\hmwkAuthorName}{Clayton Rayment} % Your name

%----------------------------------------------------------------------------------------
%	TITLE PAGE
%----------------------------------------------------------------------------------------

\title{
\vspace{2in}
\textmd{\textbf{\hmwkClass:\ \hmwkTitle}}\\
\normalsize\vspace{0.1in}\small{Due\ on\ \hmwkDueDate}\\
\vspace{3in}
}

\author{\textbf{\hmwkAuthorName}}
\date{} % Insert date here if you want it to appear below your name

%----------------------------------------------------------------------------------------

\begin{document}

\maketitle

%----------------------------------------------------------------------------------------
%	TABLE OF CONTENTS
%----------------------------------------------------------------------------------------

%\setcounter{tocdepth}{1} % Uncomment this line if you don't want subsections listed in the ToC

\newpage
\tableofcontents
\newpage

%----------------------------------------------------------------------------------------
%	PROBLEM 1
%----------------------------------------------------------------------------------------

% To have just one problem per page, simply put a \clearpage after each problem

\begin{homeworkProblem}
    Part of changing from one coordinate system to another is scaling and making things fit. E.g., suppose that you had a square with lower left corner (llc) (0,0) and upper right corner (urc) (1,1). You want to scale and center it to just fit into a rectangle with llc (0,0) and urc (2,3). The square stays a square but is probably larger or smaller. Then, these equations would do it:
    \begin{align*}
        x' &= 2x\\
        y' &= 2y + 1/2
    \end{align*}
    This question is to figure out how to make a rectangle from (0,0) to (2,3) fit into a square that is from (0,0) to (10,9).\\
    \problemAnswer{ % Answer
        We see that the limiting scale factor of this rectangle is the Y-Axis. Scaling the rectangle by 3 gives us:
        \begin{align*}
            x' &= 3x\\
            y' &= 3y\\
            RECT &= (0,0), (6,9)
        \end{align*}
        Now we need to align the rectangle to the center of the bounding box:
        \begin{align*}
            x' &= 3x + 2\\
            y' &= 3y\\
            RECT &= (2, 0), (8, 9)
        \end{align*}
    }

\end{homeworkProblem}

%----------------------------------------------------------------------------------------
%	PROBLEM 2
%----------------------------------------------------------------------------------------

\begin{homeworkProblem} % Custom section title
    Consider a pinhole camera as discussed in slide 12 of ppt presentation 1-5. Let d=2. To where does the point (1,3,-3) project? Use the equation on that slide. x/z/d should be parenthesized as x/(z/d).\\
    %--------------------------------------------
    \problemAnswer{
        We use the following relations given to us in slide 12 of presentation 1-5:\\
        \begin{align*}
            x_p &= \frac{-x}{z/d}\\
            y_p &= \frac{-y}{z/d}\\
            z_p &= d\\  
        \end{align*}
        Substituting our values for the numbers in the relation we find the following:
        \begin{align*}
            x_p &= \frac{-1}{-3/2}\\
            x_p &= \frac{2}{3}\\
            y_p &= \frac{-3}{-3/2}\\
            y_p &= 2\\
            z_p &= 2\\~\\
            P_p &= (\frac{2}{3}, 2, 2)\\
        \end{align*}

    }
    %--------------------------------------------

\end{homeworkProblem}

%----------------------------------------------------------------------------------------
%	PROBLEM 3
%----------------------------------------------------------------------------------------

\begin{homeworkProblem} % Custom section title
    If your image has only 64 different colors across the whole image, how many bits per pixel do you need for the color buffer?\\

    %--------------------------------------------
   \problemAnswer{
    Since you only need 64 distinct values, this can be provided with 6 bits per pixel:
    \begin{align*}
        log_2(64) = 6\\
    \end{align*}
   }
    %--------------------------------------------

\end{homeworkProblem}
%--------------------------------------------

%----------------------------------------------------------------------------------------
%	PROBLEM 4
%----------------------------------------------------------------------------------------

\begin{homeworkProblem}
    Extend your program from last week that displays the Starship Enterprise as follows:
    \begin{enumerate}
        \item Add 3 sliders that will rotate it around the X-axis, Y-axis, and Z-axis respectively.
        \item Do the rotations the simplest (and least efficient) way. I.e., in your javascript program, have render rotate the matrix and resend it to the GPU.\\
    \end{enumerate}
    \problemAnswer{ % Answer   
        The solution to this week's programming problem can be found at the following URL:\\
        \url{http://homepages.rpi.edu/~raymec/homeworks/hw3/ncc17013d.html}\\
        The main changes to the program this week were in the vertex shader to rotate the vertices, and in the HTML file to add sliders to the page. The javascript file was updated in order to get the values from the sliders and pass them to the vertex shader.
    }
\end{homeworkProblem}
\end{document}

%----------------------------------------------------------------------------------------