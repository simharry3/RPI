%%%%%%%%%%%%%%%%%%%%%%%%%%%%%%%%%%%%%%%%%
% Structured General Purpose Assignment
% LaTeX Template
%
% This template has been downloaded from:
% http://www.latextemplates.com
%
% Original author:
% Ted Pavlic (http://www.tedpavlic.com)
%
% Note:
% The \lipsum[#] commands throughout this template generate dummy text
% to fill the template out. These commands should all be removed when 
% writing assignment content.
%
%%%%%%%%%%%%%%%%%%%%%%%%%%%%%%%%%%%%%%%%%

%----------------------------------------------------------------------------------------
%	PACKAGES AND OTHER DOCUMENT CONFIGURATIONS
%----------------------------------------------------------------------------------------

\documentclass{article}

\usepackage{fancyhdr} % Required for custom headers
\usepackage{lastpage} % Required to determine the last page for the footer
\usepackage{extramarks} % Required for headers and footers
\usepackage{graphicx} % Required to insert images
\usepackage{lipsum} % Used for inserting dummy 'Lorem ipsum' text into the template
\usepackage{amsmath}
\usepackage{hyperref}

% Margins
\topmargin=-0.45in
\evensidemargin=0in
\oddsidemargin=0in
\textwidth=6.5in
\textheight=9.0in
\headsep=0.25in 

\linespread{1.1} % Line spacing

% Set up the header and footer
\pagestyle{fancy}
\lhead{\hmwkAuthorName} % Top left header
\chead{\hmwkClass\ (\hmwkClassInstructor\ \hmwkClassTime): \hmwkTitle} % Top center header
\rhead{\firstxmark} % Top right header
\lfoot{\lastxmark} % Bottom left footer
\cfoot{} % Bottom center footer
\rfoot{Page\ \thepage\ of\ \pageref{LastPage}} % Bottom right footer
\renewcommand\headrulewidth{0.4pt} % Size of the header rule
\renewcommand\footrulewidth{0.4pt} % Size of the footer rule

\setlength\parindent{0pt} % Removes all indentation from paragraphs

%----------------------------------------------------------------------------------------
%	DOCUMENT STRUCTURE COMMANDS
%	Skip this unless you know what you're doing
%----------------------------------------------------------------------------------------

% Header and footer for when a page split occurs within a problem environment
\newcommand{\enterProblemHeader}[1]{
\nobreak\extramarks{#1}{#1 continued on next page\ldots}\nobreak
\nobreak\extramarks{#1 (continued)}{#1 continued on next page\ldots}\nobreak
}

% Header and footer for when a page split occurs between problem environments
\newcommand{\exitProblemHeader}[1]{
\nobreak\extramarks{#1 (continued)}{#1 continued on next page\ldots}\nobreak
\nobreak\extramarks{#1}{}\nobreak
}

\setcounter{secnumdepth}{0} % Removes default section numbers
\newcounter{homeworkProblemCounter} % Creates a counter to keep track of the number of problems

\newcommand{\homeworkProblemName}{}
\newenvironment{homeworkProblem}[1][Problem \arabic{homeworkProblemCounter}]{ % Makes a new environment called homeworkProblem which takes 1 argument (custom name) but the default is "Problem #"
\stepcounter{homeworkProblemCounter} % Increase counter for number of problems
\renewcommand{\homeworkProblemName}{#1} % Assign \homeworkProblemName the name of the problem
\section{\homeworkProblemName} % Make a section in the document with the custom problem count
\enterProblemHeader{\homeworkProblemName} % Header and footer within the environment
}{
\exitProblemHeader{\homeworkProblemName} % Header and footer after the environment
}

\newcommand{\problemAnswer}[1]{ % Defines the problem answer command with the content as the only argument
\noindent\framebox[\columnwidth][c]{\begin{minipage}{0.98\columnwidth}#1\end{minipage}} % Makes the box around the problem answer and puts the content inside
}

\newcommand{\homeworkSectionName}{}
\newenvironment{homeworkSection}[1]{ % New environment for sections within homework problems, takes 1 argument - the name of the section
\renewcommand{\homeworkSectionName}{#1} % Assign \homeworkSectionName to the name of the section from the environment argument
\subsection{\homeworkSectionName} % Make a subsection with the custom name of the subsection
\enterProblemHeader{\homeworkProblemName\ [\homeworkSectionName]} % Header and footer within the environment
}{
\enterProblemHeader{\homeworkProblemName} % Header and footer after the environment
}
   
%----------------------------------------------------------------------------------------
%	NAME AND CLASS SECTION
%----------------------------------------------------------------------------------------

\newcommand{\hmwkTitle}{Assignment\ \#2} % Assignment title
\newcommand{\hmwkDueDate}{Monday,\ September\ 18,\ 2017} % Due date
\newcommand{\hmwkClass}{CSCI\ 6964} % Course/class
\newcommand{\hmwkClassTime}{4:00pm} % Class/lecture time
\newcommand{\hmwkClassInstructor}{Franklin} % Teacher/lecturer
\newcommand{\hmwkAuthorName}{Clayton Rayment} % Your name

%----------------------------------------------------------------------------------------
%	TITLE PAGE
%----------------------------------------------------------------------------------------

\title{
\vspace{2in}
\textmd{\textbf{\hmwkClass:\ \hmwkTitle}}\\
\normalsize\vspace{0.1in}\small{Due\ on\ \hmwkDueDate}\\
\vspace{3in}
}

\author{\textbf{\hmwkAuthorName}}
\date{} % Insert date here if you want it to appear below your name

%----------------------------------------------------------------------------------------

\begin{document}

\maketitle

%----------------------------------------------------------------------------------------
%	TABLE OF CONTENTS
%----------------------------------------------------------------------------------------

%\setcounter{tocdepth}{1} % Uncomment this line if you don't want subsections listed in the ToC

\newpage
\tableofcontents
\newpage

%----------------------------------------------------------------------------------------
%	PROBLEM 1
%----------------------------------------------------------------------------------------

% To have just one problem per page, simply put a \clearpage after each problem

\begin{homeworkProblem}
    Which RPI grad was the technical person in the founding group of NVidia?\\~\\
    \problemAnswer{ % Answer
        Curtis Priem was an Electrical Engineering major at RPI in 1982, and co-founded NVidia with Jen-Hsun Huang and Chris Malachowsky in 1993.
    }

\end{homeworkProblem}

%----------------------------------------------------------------------------------------
%	PROBLEM 2
%----------------------------------------------------------------------------------------

\begin{homeworkProblem} % Custom section title
    What hardware component had to get much cheaper in order to make frame buffers possible?\\~\\

    %--------------------------------------------
    \problemAnswer{
        RAM had to get much cheaper before framebuffers were feasable. At the time (1974), due to expensive memory, the first commercial framebuffer cost around \$15,000, and only suppored resolutions up to 512x512 in 8-bit greyscale.
    }
    %--------------------------------------------

\end{homeworkProblem}

%----------------------------------------------------------------------------------------
%	PROBLEM 3
%----------------------------------------------------------------------------------------

\begin{homeworkProblem} % Custom section title
    The process of computer graphics has always been enabled by new hardware. One newish toy is Google Cardboard. Pretend that I've never heard of it, and write 100 words or so summarizing it and giving specific details.\\

    %--------------------------------------------
   \problemAnswer{
    Google Cardboard is an open-source VR platform, which is supported by most Android and iOS phones. The actual viewer itself can be constructed extremely cheaply using common materials, or can be purchased from several manufacturers since the specifications are open-source. Once the head unit is constructed, the user's phone can be placed inside the viewer. The screen will be partitioned into two parts, each one being viewed by the right or left eye to give the illusion of stereoscopic vision. With gyro/accelerometer head tracking, a user can experience immersion as the scene will be updated based on head position.
   }
    %--------------------------------------------

\end{homeworkProblem}
%--------------------------------------------

%----------------------------------------------------------------------------------------
%	PROBLEM 4
%----------------------------------------------------------------------------------------

\begin{homeworkProblem}
    Modify last week's program to display the spaceship NCC1701 using the vertex data file given by \url{https://wrf.ecse.rpi.edu/Teaching/graphics-f2017/files/ncc1701b.data}.\\

    \problemAnswer{ % Answer        
    }
\end{homeworkProblem}

%----------------------------------------------------------------------------------------

%----------------------------------------------------------------------------------------
%	PROBLEM 5
%----------------------------------------------------------------------------------------

\begin{homeworkProblem}
    In real life, light bounces from object to object, on its way from the light source to the viewer. However the OpenGL pipeline processes objects independently, and does not allow that (except that an object can hide another object). Why?
    
    \problemAnswer{ % Answer
        The calculation of light reflections between objects is called raytracing. Raytracing is an extremely expensive operation, and with the resources available to OpenGL, it becomes impossible to do such raytracing while keeping the scene in realtime.
    }
\end{homeworkProblem}
    
%----------------------------------------------------------------------------------------


%----------------------------------------------------------------------------------------
%	PROBLEM 6
%----------------------------------------------------------------------------------------

\begin{homeworkProblem}
    According to the tristimulus model, our eyes have three types of color receptors (cones). However several species and some rare human females have four types of cones. They are called tetrachromats.

    \begin{homeworkSection}{(a)}
        Name some such animals.\\
        \problemAnswer{
            \begin{itemize}
                \item Reindeer
                \item Zebra Finch
                \item Goldfish
                \item Bonus: Mantis Shrimp (16 cone pigments!)
            \end{itemize}
        }
    \end{homeworkSection}

    \begin{homeworkSection}{(a)}
        For humans, why is it (probably; opinions differ) only females?\\
        \problemAnswer{
            One possible explanation is that two cone cell pigment genes are on the X chromosome, so women (who have two x-chromosomes) could possibly recieve 4 different pigment types.\\
        }
    \end{homeworkSection}
\end{homeworkProblem}
    
%----------------------------------------------------------------------------------------

\end{document}
